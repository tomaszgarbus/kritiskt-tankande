\section{Course book -- Chapter 1}

\subsection{Definitions}
\begin{itemize}
	\item \textbf{reconstructing an argument} -- the process of identifying
	the premises and conclusions in a piece of argumentative writing
	\item \textbf{evaluating an argument} -- the process of determining
	whether an argument is a good argument
	\item \textbf{argument} -- a sequence of propositions intended to
	establish the truth of one of the propositions. The components of an
	argument are its premises and conclusion.
	\item \textbf{conclusion} -- what an argument is intended to establish;
	the point of an argument; the proposition an argument is supposed to
	support
	\item \textbf{premise} -- a part of an argument that is supposed to
	help establish the argument's conclusion
	\item \textbf{argument analysis} -- the process of interpreting
	(reconstructing) and evaluating an argument
	\item \textbf{rhetorical power} -- the power to persuade or convince.
	Arguments, as well as people, can have rhetorical power. Contrast with
	\textit{rational strength} and \textit{literary merit}.
	\item \textbf{rational strength} -- the degree to which something
	provides good reason to believe something
	\item \textbf{literary merit} -- the quality of a piece of writing
	determined primarily by whether it is well-written, original,
	well-organized, and interesting.
	\item \textbf{argument stopper} -- a response to an argument that has
	the effect of cutting off discussion and preventing careful argument
	analysis
\end{itemize}

\subsection{Notes}

Steps of argument analysis:
\begin{enumerate}
	\item Reconstruct the argument
	\item Evaluate the argument
\end{enumerate}

Impediments to good reasoning:
\begin{enumerate}
	\item Lack of adequate vocabulary
	\item The desire to be tolerant and open-minded
	\item Misunderstanding the point of argument analysis
	\item Misconceptions about truth and rationality
	\item The use of argument stoppers
\end{enumerate}
