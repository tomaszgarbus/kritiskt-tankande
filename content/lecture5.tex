%&/()%&/()%&/()%&/()%&/()%&/()%&/()%&/()%&/()%&/()%&/()%&/()%&/()%&/()%&/()%&/()

\section{Lecture 5: Reconstructing Arguments}

\subsection{Argumentation Analysis}

\begin{enumerate}
    \item Reconstruct the argument
    \item Evaluate the argument
\end{enumerate}

\subsection{Reconstruction}

\textit{All men are mortal. Socrates is a man. Therefore
Socrates is mortal.}

\begin{enumerate}
    \item All men are mortal.
    \item \underline{Socrates is a man.}
    \item Socrates is mortal.
\end{enumerate}

\subsection{Steelmanning}

Putting forward the best (strongest) version of one's
opponent's argument before criticizing it.

\subsection{Principle of Charity (Feldman)}

When reconstructing an argument, try to formulate a reconstruction that is
well-formed, has reasonable premises and is undefeated.

\subsection{Argument reconstruction}

\begin{enumerate}
    \item Decide if there is an argument.
    \item Reconstruct the argument.
    \begin{enumerate}
        \item Identify the conclusion.
        \item Identify explicit premises.
        \item Check if well-formed, if not:
        \item Add implicit premises.
    \end{enumerate}
    \item Fine-tune the reconstruction.
    \begin{enumerate}
        \item Clarify wording.
        \item Add justifications, clarify implicit and explicit premises.
    \end{enumerate}
\end{enumerate}

\subsection{Identify the conclusion(s)}

\textbf{Conclusion:} the proposition an argument is supposed to establish

Look for conclusion indicators before the conclusion (\textit{therefore},
\textit{I conclude}...) or after (\textit{after all}, \textit{because},
\textit{this is established by}...).

Note: it is possible that the conclusion is not straightforwardly stated
in the argument.

\textit{Philosophy is difficult to study. This is because philosophy
requires thinking for oneself rather than simply learning information.
In addition, philosophy is unlike any other discipline.}

\textit{Taxation is theft.}

\textbf{Intermediate conclusion:} A conclusion that becomes a premise in a
wider argument.

\begin{enumerate}
    \item If the libertarians win the election, taxes on savings will be lower.
    \item \underline{The libertarians will win the election.}
    \item Taxes on savings will be lower.
    \item \underline{If taxes on savings will be lower, you should save more.}
    \item You should save more.
\end{enumerate}

\subsection{Identify the explicit premises}

\begin{itemize}
    \item Look for premise indicators (\textit{follows from the fact that...})
    \item Ask yourself: what are the reasons given for the conclusion?
    \item Premises can be: specific statements (\textit{Socrates is a man}),
        universal generalizations (\textit{all men are mortal}),
        non-universal generalisations (\textit{most politicians are liars}).
\end{itemize}

\subsection{Is the argument well-formed?}

Well-formed arguments:

\begin{itemize}
    \item valid arguments
    \item cogent arguments
\end{itemize}

\subsection{Add implicit premises (to make it well-formed)}

\textbf{Implicit premise:} A premise that is not stated in the argument, but
which we need to make the argument well-formed.

\subsection{Principle of faithfulness}

Add implicit premises that are consistent with the intensions of the author
of the argument.

\subsection{Principle of charity for implicit premises}

Add implicit premises that are reasonable to accept rather than implicit
premises that are obviously false.

\subsection{Clarify wording}

Try not to move between different wordings of the same idea.

\subsection{Add justifications, clarify explicit and implicit premises}

\begin{itemize}
    \item Each conclusion (including intermediate conclusions) should be
        followed by the number of each premise that supports it in
        parentheses.
    \item Each premise should be followed by (EP) or (IP) depending on if
        it's explicit or implicit.
\end{itemize}

\subsection{Some additional dangers}

\begin{itemize}
    \item missing premises
    \item including unnecessary premises (only include explicit premises if
        they make the argument stronger)
    \item improper wording
\end{itemize}

%&/()%&/()%&/()%&/()%&/()%&/()%&/()%&/()%&/()%&/()%&/()%&/()%&/()%&/()%&/()%&/()