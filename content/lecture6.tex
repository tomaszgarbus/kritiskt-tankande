%&/()%&/()%&/()%&/()%&/()%&/()%&/()%&/()%&/()%&/()%&/()%&/()%&/()%&/()%&/()%&/()

\section{Lecture 6: Evaluating Arguments}

\subsection{Homework: Reconstruct arguments}

\textit{Reconstruct the following arguments. State whether each line
of your reconstruction is an implicit premise (by writing “EP” after),
an implicit premise (by writing “IP” after) or a conclusion (by listing the
premises that support it).}

\subsubsection{1}

\textit{Sarah should win the “Best Author” award. She has written a large
amount of books, and her writing style is engaging. Anyone who has these
qualities should definitely win.}

\begin{enumerate}
    \item (EP) Sarah has written a large amount of books.
    \item (EP) Sarah's style of writing is engaging.
    \item (EP) Anyone who has written a large amount of books and
        has an engaging style of writing should win the "Best Author" award.
    \item (C) Therefore, Sarah should win the "Best Author" award.
\end{enumerate}

%&/()%&/()%&/()%&/()%&/()%&/()%&/()%&/()%&/()%&/()%&/()%&/()%&/()%&/()%&/()%&/()
\subsubsection{2}

\textit{Switzerland should not adopt the Euro because doing so would
require ceding significant monetary sovereignty, and ceding monetary
sovereignty undermines a state’s ability to respond independently
to economic shocks. Switzerland, being a small but highly open economy,
needs freedom to set its own monetary policy during crises. Therefore,
Switzerland should remain outside the Eurozone.}

\begin{enumerate}
    \item (EP) Adopting the Euro requires ceding monetary sovereignty.
    \item (EP) Ceding monetary sovereignty undermines the state's pliability
        to economic crises.
    \item (EP) Switzerland is a small but highly open economy.
    \item (IP) Small but open economies need freedom to set their own
        monetary policies during crises.
    \item (C) Switzerland needs the freedom to set its own monetary policy
        during crises.
    \item (IP) If a country needs the freedom to set its own monetary policies
        to handle economic crises, it should not adopt the Euro.
    \item Switzerland should not adopt the Euro.
\end{enumerate}

\subsubsection{3}
%&/()%&/()%&/()%&/()%&/()%&/()%&/()%&/()%&/()%&/()%&/()%&/()%&/()%&/()%&/()%&/()
\textit{I oppose the use of AI in weapons systems. The current state of
the technology is simply not good enough to distinguish between soldiers
(who are legitimate targets in a war) and non-combatants (who are not
legitimate targets). If these systems are deployed, they are therefore
likely to lead to mass civilian casualties. Even if we could improve
them to the point that they were as good as (if not better) than human
soldiers, however, there is a deeper issue here. Who is responsible for
the deaths caused by an AI weapons system? It can’t be the military
commanders who order them to be sent out, since they won’t have a
sufficient understanding of what they’re going to do. It also can’t be
the programmers, as they wouldn’t be able to predict or control how the
systems would behave once deployed. So we must conclude that nobody is
responsible. That sort of situation – where nobody is responsible for
life and death decisions, is something to be avoided at all costs.
It’s a vision of a dystopian future.}

\begin{enumerate}
    \item (EP) Non-combatants are not legitimate targets.
    \item (EP) The current state of AI is not good enough to distinguish
        between soldiers and non-combatants.
    \item (C) Therefore these systems are likely to cause mass non-combatant
        casualties.
    \item (EP) Military commanders don't have a sufficient understanding
        of what AI systems are going to do.
    \item (IP) If one doesn't have sufficient understanding of system's
        future behaviour, one cannot be responsible for that system.
    \item (C) Therefore, military commanders cannot be responsible for
        the deaths caused by AI.
    \item (EP) The programmers are not able to predict or control the
        systems' behaviour once deployed.
    \item (IP) If one can't predict or control a system's behaviour,
        once can't be responsible for that system.
    \item (C) Therefore, the programmers cannot be responsible for the
        deaths caused by AI.
    \item (IP) If neither programmers nor military commanders can be
        responsible for the deaths caused by AI, then nobody can.
    \item (C) Therefore, nobody can be responsible for the deaths
        caused by AI.
    \item (EP) Situations where nobody is responsible for life and
        death decisions must be avoided at all costs.
    \item Therefore, AI should not be used in weapons systems.
\end{enumerate}

\subsection{Argumentation analysis}

\begin{enumerate}
    \item Reconstruct the argument.
    \item \textbf{Evaluate the argument.}
\end{enumerate}

%&/()%&/()%&/()%&/()%&/()%&/()%&/()%&/()%&/()%&/()%&/()%&/()%&/()%&/()%&/()%&/()
\subsection{Evaluate the argument}
(0) Is the argument well-formed -- hopefully you have been
        able to make it well-formed.
\begin{enumerate}
    \item Go through each premise, stating whether (and why) you accept
        the premise, you reject the premise, or you suspend judgement about
        the premise.
    \item If you accept all the premises, and the argument is well-formed,
        the argument is strong (for you).
    \item If you reject one or more of the premises and/or suspend judgement
        on one or more of the premises, the argument is weak (for you).
\end{enumerate}

\subsection{Feldman's rules of evaluation}

\begin{itemize}
    \item Don't criticize an argument by denying its conclusion.
    \item Don't accept an argument simply because you believe the conclusion.
    \item Direct criticisms at individual premises.
    \item Make your criticisms of premises substantial.
    \item Don't accept competing arguments.
    \item Don't object to intermediate conclusions of compound arguments.
\end{itemize}

\subsection{Evaluating premises}

\subsubsection{Specific factual claims}

\begin{itemize}
    \item It will rain in Stockholm tonight.
    \item The freezing point of water is 0 degrees C.
\end{itemize}

\subsubsection{Generalizations}

\begin{itemize}
    \item All politicians are male.
    \item Almost all Swedes speak English.
\end{itemize}

\subsubsection{Compound sentences}

\begin{itemize}
    \item It will rain tomorrow or it will snow tomorrow.
    \item All universities in Sweden are outside Stockholm
        or Stockholm has a university.
\end{itemize}

\subsection{Evaluating premises: Specific factual claims}

\begin{itemize}
    \item Use observation, reading, memory.
    \item May need to rely on expert testimony.
\end{itemize}

\subsection{Evaluating premises: Generalizations}

\begin{itemize}
    \item Provide counter example (only for universal generalization)
\end{itemize}

\subsection{Evaluating premises: Compound sentences}

\begin{itemize}
    \item You can criticize conjunctions (P and Q) by providing evidence
        against either P or Q.
    \item You can criticize disjunctions (P or Q) by showing that they
        involve a false dilemma.
\end{itemize}

\subsection{Ethical principles}

\begin{itemize}
    \item We can criticize ethical principles by using thought experiments
        to show that the principle has counterintuitive implications.
\end{itemize}

\subsection{Reconstruct and evaluate}

\textit{When deciding on criminal sentencing, we should always try
to reduce crime. Increasing the sentences given to terrorists will reduce that
sort of crime. So we should increase sentences for terrorists.}

\begin{enumerate}
    \item (EP) When deciding on criminal sentencing, we should always try to
        reduce crime.
    \item (EP) Increasing the sentences given to terrorists will
        reduce that sort of crime.
    \item We should increase sentences for terrorists.
\end{enumerate}

First premise is an ethical principle.