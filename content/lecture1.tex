%%%%%%%%%%%%%%%%%%%%%%%%%%%%%%%%%%%%%%%%%%%%%%%%%%%%%%%%%%%%%%%%%%%%%%%%%%%%%%%

\section{Lecture 1: Introduction}

\subsection{Socrates (c. 479-388 BC)}

\textit{The unexamined life is not worth living.}

\subsection{Marco Rubio, US Presidential Hopeful (2015)}

\textit{We need more welders and less philosophers.}

\subsection{Science}

Experiment $\rightarrow$ Observation $\rightarrow$ Theory

\subsection{Moral philosophy}

Thought experiment $\rightarrow$ Intuition $\rightarrow$ Moral theory

\textbf{Utilitarianism} says that the right action is one that maximizes
welfare.

\subsection{Implications for Policy}

\begin{enumerate}
    \item Which medical research should we fund?
    \item Should we go to war?
    \item How should we balance security and privacy in counterterrorism?
    \item Who should have the rights over new technologies?
    \item Should we tax people more to fund public services?
\end{enumerate}

Even if philosophy cannot give us certainty, it can help us see how
complicated certain issues are.

Philosophy is (primarily) a skill, rather than a body of knowledge.

\subsection{Take-Away Points}

\begin{itemize}
    \item Philosophy has a method. We need to practice to become good in
        that method.
    \item Don't expect easy answers in philosophy. This doesn't mean that
        it's worthless.
    \item Using philosophy in the real world needs to be done with caution.
\end{itemize}

\subsection{What to do before each lecture}
\begin{enumerate}
    \item Complete any readings.
    \item Complete the online quiz on the topic of that lecture.
    \item Complete questions on the topic of the previous lecture.
\end{enumerate}

\subsection{Exam}

28 October 18:00 -- 22:00

\subsection{Points of contact}

isaac.taylor@philosophy.su.se (for academic questions)

info@philosophy.su.se (for organizational questions)

\subsection{Argument (noun)}

\begin{enumerate}
    \item (...)
    \item a coherent series of reasons, statements, or facts intended to
        support or establish a point of view
    \item an angry quarrel or disagreement
\end{enumerate}

\begin{itemize}
    \item I did a rain dance this morning. Therefore, it is raining now.
    \item We should save the most lives. Imposing a lockdown will save the
        most lives. So we should impose a lockdown.
    \item Taxation is theft.
    \item The weather forecast said that it is raining. I hear the sound of
        rain on the roof. Therefore it is raining now.
\end{itemize}

In order to know whether an argument is strong or weak, we need to do
argumentation analysis (a skill to be developed in this course).

\begin{itemize}
    \item \textbf{strong argument} gives us rational reason to believe its
        conclusion
    \item \textbf{weak argument} does not give us rational reason to believe
        its conclusion
\end{itemize}

\subsection{An argument of Rene Descartes}

\textit{Common sense is the most fairly distributed thing in the world,
for each one thinks he is so well-endowed with it that even those who are
hardest to satisfy in all other matters are not in the habit of desiring
more of it than they already have.}

Is this a strong argument? Why or why not?

\subsection{Weak arguments}

Formal fallacy example:

\textit{All self-made millionaires have a good work ethic. Simon has a good
work ethic. So Simon must be a self-made millionaire.}

Informal fallacy example:

\textit{We should not adopt a smoking ban in public places. The Nazis were
the first to introduce such a ban and we should never adopt a policy that
Nazis adopted.}

\subsection{Two ways in which an argument can be weak}

\begin{enumerate}
    \item \textbf{It is not well-formed.} An argument is ill-formed when the
        logical structure of the argument means that there is an insufficient
        link between the premises and conclusion. Arguments of this sort
        involve \textbf{formal fallacies}.
    \item \textbf{It has implausible premises.} Arguments of this sort involve
        \textbf{informal fallacies}.
\end{enumerate}

\subsection{Red Herring}

Including an irrelevant premise that is relied on in arguing for the
conclusion.

\textit{The judge should rule against the charge of false accounting
against the CEO of Utilicorp. \ul{The CEO is very popular with
shareholders, and presides over an extremely healthy company.}}

\textit{If you want to see your investment in dollars grow, trust our
investment advisors at Inside Traders, Inc. ... Our customers stick with
us and we have become one of the most successful investment advisory
firms in the nation.}

Compare the red herring fallacy with the dead cat strategy:

\textit{That is because there is one thing that is absolutely certain
about throwing a dead cat on the dining room table – and I don't mean that
people will be outraged, alarmed, disgusted. That is true, but irrelevant.
The key point, says my Australian friend, is that everyone will shout
"Jeez, mate, there's a dead cat on the table!"; in other words they will
be talking about the dead cat, the thing you want them to talk about, and
they will not be talking about the issue that has been causing you so much
grief.}

\subsection{Equivocation}

Improperly moving between two different meanings of the same word in an
argument.

\textit{Only the President can make the decision to ban people from Stockholm
Golf Club. Donald Trump is the President. Therefore, only Donald Trump can
make the decision to ban people from Stockholm Golf Club.}

\textit{The Western liberal idea that people have the same rights in all
times and places is false. People in different countries have different
rights.}

\subsection{Straw Man}

Mischaracterising one's opponents argument to make it look weaker than it is.

\textit{How can you support this assisted dying bill? You can't tell me that
it is moral to encourage relatives to get vulnerable people to end their lives
as soon as they become an inconvenience or, worse still, when they want to get
a hold on inheritance. This is obscene!}

%%%%%%%%%%%%%%%%%%%%%%%%%%%%%%%%%%%%%%%%%%%%%%%%%%%%%%%%%%%%%%%%%%%%%%%%%%%%%%%