%%%%%#####%%%%%#####%%%%%#####%%%%%#####%%%%%#####%%%%%#####%%%%%#####%%%%%#####

\section{Lecture 3: Valid Arguments}

\subsection{Argumentation Analysis}

\begin{enumerate}
    \item Reconstruct the argument
    \item \textbf{Evaluate the argument}
\end{enumerate}

We want to use argumentation analysis to identify
\textbf{strong arguments}: arguments that give us a rational
reason to believe the conclusion.

\subsection{Two sorts of weakness}

\textit{The President of the US must be 100 years old or more. Donald Trump
is 100 years old or more. Therefore DT is President of the US.}

\begin{enumerate}
    \item The President of the US must be 100 yo or more.
    \item DT is 100 yo or more.
    \item DT is President of the US.
\end{enumerate}

Weakness 1: premises are not plausible.

Weakness 2: even if the premises are true, the conclusion would not follow
(the argument is \textbf{invalid}).

A valid argument can have a false conclusion.

An invalid argument can have a true conclusion.

\subsection{Deductively strong arguments}

\begin{itemize}
    \item Valid: the truth of the premises guarantees the truth of the
        conclusion.
    \item Justified premises: it is reasonable to believe all the premises.
\end{itemize}

\subsection{Patterns of arguments}

\subsubsection{Sentential/Propositional logic}

Represents propositions (complete sentences) with letters (P, Q, R, ...).

\begin{enumerate}
    \item Either my son is on the phone or my wife is on the phone.
    \item My son is not on the phone.
    \item My wife is on the phone.
\end{enumerate}

\begin{enumerate}
    \item Either P or Q
    \item not P
    \item Q
\end{enumerate}

\subsubsection{Predicate logic}

\begin{enumerate}
    \item All men are mortal.
    \item Socrates is a man.
    \item Socrates is mortal.
\end{enumerate}

\begin{enumerate}
    \item All As are Bs.
    \item x is an A.
    \item x is a B.
\end{enumerate}

\subsection{Valid argument patterns in propositional logic}

\subsubsection{Argument by elimination}

\begin{enumerate}
    \item Either P or Q
    \item not P
    \item Q
\end{enumerate}

\subsubsection{Simplification}

\begin{enumerate}
    \item P and Q
    \item P
\end{enumerate}

\subsubsection{Affirming the antecedent}

\begin{enumerate}
    \item If P then Q
    \item P
    \item Q
\end{enumerate}

\subsubsection{Denying the consequent}

\begin{enumerate}
    \item If P then Q
    \item not Q
    \item not P
\end{enumerate}

\subsection{Invalid argument patterns (formal fallacies)
in propositional logic}

\subsubsection{Denying the antecedent}

\begin{enumerate}
    \item If P then Q
    \item not P
    \item not Q
\end{enumerate}

\subsubsection{Affirming the consequent}

\begin{enumerate}
    \item If P then Q
    \item Q
    \item P
\end{enumerate}

\subsection{Valid argument patterns in predicate logic}

Patterns provable in FOPL.

\subsection{Invalid argument patterns in predicate logic}

Patterns disprovable in FOPL.

\subsection{Lexicons}

Especially when we are dealing with more complex arguments, it is useful to
write down what all the letters represent. This is called a \textit{lexicon}.

\subsection{Formal vs informal fallacies}

\textbf{Formal fallacies} make an argument \textbf{invalid.}

\textbf{Informal fallacies} make it \textbf{reasonable to reject a premise.}

%%%%%#####%%%%%#####%%%%%#####%%%%%#####%%%%%#####%%%%%#####%%%%%#####%%%%%#####