\section{Lecture 7: Generalizations and analogies}

\subsection{Evaluating Generalizations: Universal
Generalizations ("All As are Bs")}

\begin{itemize}
    \item You can reject a universal generalization by providing
        just one counter-example
    \item This also works for generalizations of the form "No As
        are Bs"
\end{itemize}

\subsection{Evaluating generalizations}

\begin{itemize}
    \item Insufficient sample
    \item Unrepresentative sample
\end{itemize}

\subsection{Informal fallacies: Post Hoc Ergo Propter Hoc}

Inferring that one event caused another event because it occurred
before another.

\subsection{Informal fallacies: Mistaking correlation for cause}

Unjustifiably inferring a causal connection between two things because they
are correlated.

\subsection{Evaluating arguments by analogy}

\begin{enumerate}
    \item Set out the analogy and the case it is supposed to represent.
    \item List all the relevant similarities.
    \item List all the relevant differences.
    \item On the basis of 1-3, decide if the analogy is strong
        (many relevant similarities), weak (few relevant similarities)
        or false (relevant dissimiliarities).
\end{enumerate}