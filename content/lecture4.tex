%####%####%####%####%####%####%####%####%####%####%####%####%####%####%####%####
\section{Lecture 4: Cogent Arguments}

\subsection{Validity and cogency}

\textit{cogent: Appealing to the intellect or powers of reasoning;
convincing: synonym: valid}

Example of a cogent argument:

\begin{enumerate}
    \item Most Swedes speak English.
    \item Åsa is a Swede.
    \item Åsa speaks English.
\end{enumerate}

Cogent argument pattern:

\begin{enumerate}
    \item Most As are Bs.
    \item x is an A.
    \item x is a B.
\end{enumerate}

\textbf{Cogent Arguments}: If the premises are true, the conclusion
is more likely than not in virtue of this (although never guaranteed).

\subsection{Non-cogent argument patterns}

\subsubsection{Denying the antecedent}

Pattern: 
\begin{enumerate}
    \item Most As are Bs.
    \item x is not an A.
    \item x is not a B.
\end{enumerate}

Example:
\begin{enumerate}
    \item Most footballers are well-paid.
    \item Bill is not a footballer.
    \item Bill is not well-paid.
\end{enumerate}

\subsubsection{Affirming the consequent}

Pattern: 
\begin{enumerate}
    \item Most As are Bs.
    \item x is a B.
    \item x is an A.
\end{enumerate}

Example:
\begin{enumerate}
    \item Most bankers are well-paid.
    \item Brenda is well-paid.
    \item Brenda is a banker.
\end{enumerate}

\subsubsection{More patterns}

Very very weakly cogent:
\begin{enumerate}
    \item Some As are Bs.
    \item x is an A.
    \item x is a B.
\end{enumerate}

Valid argument:
\begin{enumerate}
    \item No As are Bs.
    \item x is an A.
    \item x is a B.
\end{enumerate}

\subsection{A note on "Background knowledge"}

Example 1:
\begin{enumerate}
    \item Most Swedes speaks English.
    \item Ulf Kristersson speaks English.
\end{enumerate}

Example 2:
\begin{enumerate}
    \item All men are mortal.
    \item Socrates is mortal.
\end{enumerate}

\subsection{Strong argument}
\textbf{Strong argument: } an argument that gives us a rational reason to
believe the conclusion.

\subsection{Defeat}
A cogent argument can be defeated if we have additional evidence.

\subsection{Deductively strong arguments}

\begin{itemize}
    \item Valid: the truth of the premises guarantees the truth of the
        conclusion.
    \item Justified premises: it is reasonable to believe all the premises.
\end{itemize}

\subsection{Inductively strong argument}
An argument that is cogent, has reasonable premises, and is not defeated by
one's total evidence

\subsection{Overview: evaluating arguments}

Well-formed arguments: valid arguments and cogent arguments (these two sets
are distinct!)

\subsection{Ambiguity}
A word is ambiguous if it has more than one meaning.

\begin{itemize}
    \item Examples of ambiguous terms: bank, right, rock, stamp.
    \item Many arguments will use ambiguous terms. This is only a problem if
        it relies on ambiguity to gain rhetorical power.
\end{itemize}

\subsection{Vagueness}

The meaning of a word is vague if it is imprecise and admits of borderline
cases.

\begin{itemize}
    \item bald
    \item pile
    \item rich
\end{itemize}

%####%####%####%####%####%####%####%####%####%####%####%####%####%####%####%####